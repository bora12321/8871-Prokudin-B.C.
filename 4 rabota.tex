\documentclass[11pt,a4paper]{article}
\usepackage[T2A]{fontenc}
\usepackage[utf8]{inputenc}
\usepackage[english, russian]{babel}
\usepackage{graphicx}
\usepackage{tikz}
\usetikzlibrary{calc}
\author{Прокудин Б.С.}
\title{Практическая работа №4}

\begin{document}
\maketitle
\centering

 Поворот молекулы метана вокруг оси Y на $\frac{2\pi}{3}$ и после вокруг оси Z на $\frac{\pi}{6}$ 

\begin{tikzpicture}

\begin{scope}[xscale=5,yscale=5] 

\draw [rounded corners=4pt,color=white,ball color=blue,smooth]
(0,0) circle (0.3) node {$H_1$};%1

\draw [rounded corners=4pt,color=white,ball color=blue,smooth]
(-0.433,0.866) circle (0.3) node {$H_2$};%2

\draw [rounded corners=4pt,color=white,ball color=blue,smooth]
(-0.9873,0.0248) circle (0.3) node {$H_4$};%4

\draw [rounded corners=4pt,color=white,ball color=red,smooth]
(-0.5174,0.331) circle (0.4) node {C};%C

\draw [rounded corners=4pt,color=white,ball color=blue,smooth]
(-0.6495,0.433) circle (0.3) node {$H_3$};%3


\end{scope} 
\end{tikzpicture}
\end{document}

Начальные координаты атомов 
% Н2_0 (1,0,0) Н3_0 (0.5,0.866,0) Н4_0(0.5,0.2887,0.8165)
% C_0 (0.5,0.2887,0.204)
После 1 поворота
% Н2_у (-0.5,0,0.866)  Н3_у (-0.25,0.866,0.433)    Н4_у (-0.9734,0.2887,0.02475) 
% C_у   (-0.43083,0.2887,0.33095)
После двух поворотов
% Н2_у,z (-4.333,-0.25,0.866)    Н3_у,z (-0.6495,0.625,0.433)   Н4_у,z (-0.9873,-0.2367,0.0248) 
% C2_у,z (-0.5174,0.0346,0.331)
Координата Х убрана



