\documentclass[a4paper,11pt]{article}
\usepackage[T2A]{fontenc}
\usepackage[utf8]{inputenc}
\usepackage[english, russian] {babel}
\usepackage{tikz}
\usetikzlibrary{calc} % для мат вычислений(но это не точно)
\usepackage[european,cuteinductors, smartlabels]{circuitikz}
\title{Практическая работа № 2}

\author{Прокудин Борис Сергеевич}

% Start the document
\begin{document}
\maketitle
\begin{tikzpicture} [scale=0.8]
\newcommand{\alfa } {30}
\newcommand{\teta } {65}
\newcommand {\D } {10}
 % тетрадные клеточки
 \draw[help lines] (-3,-3) grid (20,12);
 % ось X
\draw[thin,->,>=stealth'] (0,0) - - (14,0) node[below] {$X_1$};
%ось X2
\draw[thin,->,>=stealth'] (0,0) - - ({12*cos(\teta) },{12*sin(\teta) }) node[left] {$X_2$};
% вектор х
\draw[red,->,>=stealth'] (0,0) - - ({\D*cos(\alfa) },{\D*sin(\alfa) }) node[at end, above right] {$\vec{x}$};
%перпендикуляр к оси Х1
\draw[blue, dashed] ({\D*cos(\alfa) },{\D*sin(\alfa) }) - - ++({cos(90)},{-\D*sin(\alfa)}) node[below] {$x_1$};
%контрвариантная линия к X2 ({sin(180)=0)
\draw[green,  dashed] ({\D*cos(\alfa) },{\D*sin(\alfa) }) - - ({\D*sin(\alfa)*cos(\teta)/sin(\teta)},{\D*sin(\alfa)}) node[left] {$x^2$};
%перпендикуляр к оси X2
\draw[blue, dashed]  ({\D*cos(\alfa) },{\D*sin(\alfa) }) - - +({-(\D*sin(\teta-\alfa)*cos(90-\teta)},{(\D*sin(\teta-\alfa)*sin(90-\teta)}) node[left] {$x_2$};
% контрвариантная к оси Х1
\draw[green, dashed]  ({\D*cos(\alfa) },{\D*sin(\alfa) }) - - +({-cos(\teta)*\D*sin(\alfa)/sin(\teta)},{-\D*sin(\alfa)}) node[at end,below] {$x^1$};




\end{tikzpicture}

%\begin{tikzpicture}[scale=0.5]<br>
 % \draw[help lines] (0,0) grid (4,3);<br>
  %\fill [red]   ($2*(1,1)$)              circle (2pt);<br>
  %\fill [green] (${1+1}*(1,.5)$)         circle (2pt);<br>
 % \fill [blue]  ($cos(0)*sin(90)*(1,1)$) circle (2pt);<br>
 % \fill [black] (${3*(4-3)}*(1,0.5)$)    circle (2pt);<br>
%\end{tikzpicture}



%\draw[green] (0,0) - - ++({cos(90)},{\D*sin(\alfa)});
%\draw[green] (0,0) - - ++(\D*sin\alfa*ctg\teta,\D*sin\alfa);


 %\draw[red] ({\D*cos(\alfa) },{\D*sin(\alfa) }) - - ++({-\D*cos(\alfa)},{sin(180)}); непомню к чему это(((

\end{document}