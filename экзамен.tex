\documentclass[14pt]{beamer}
\usepackage[T2A]{fontenc}
\usepackage[utf8]{inputenc}
\usepackage[english,russian]{babel}
\usepackage{tikz}
\usepackage[european,cuteinductors,smartlabels]{circuitikz}

\usepackage{amssymb,amsfonts,amsmath,mathtext}
\usepackage{amssymb}
\usepackage{cite,enumerate,float,indentfirst}
\usepackage{cancel}
\usepackage{csquotes}
\newcommand{\quotes}[1]{``#1''}

\graphicspath{{images/}}

%%%%%%% ОФОРМЛЕНИЕ СЛАЙДОВ %%%%%%%%%%%%%%%%%
\usetheme{Pittsburgh}
\usecolortheme{whale} %whale

%\usetheme{Antibes} % элемент оформления (название презентации на каждом слайде)
%\usecolortheme{seahorse}

%%%%%%% ЗАПОЛНЕНИЕ НИЖНЕЙ СТРОКИ %%%%%%%%%%%%
\setbeamercolor{footline}{fg=blue} % цвет нижней надписи
\setbeamertemplate{footline}{
\leavevmode%
\hbox{%
\begin{beamercolorbox}[wd=.333333\paperwidth,ht=2.25ex,dp=1ex,center]{}%
Прокудин Б.С.
\end{beamercolorbox}%
\begin{beamercolorbox}[wd=.333333\paperwidth,ht=2.25ex,dp=1ex,center]{}%
Санкт-Петербург, 2019
\end{beamercolorbox}%
\begin{beamercolorbox}[wd=.333333\paperwidth,ht=2.25ex,dp=1ex,right]{}%
Стр. \insertframenumber{} из \inserttotalframenumber \hspace*{2ex}
\end{beamercolorbox}}%
\vskip0pt%
}
%%%%%%% КОНЕЦ ЗАПОЛНЕНИЕ НИЖНЕЙ СТРОКИ %%%%%%%%%


\newcommand{\itemi}{\item[\checkmark]}

\usefonttheme[onlymath]{serif} % в формулах использовать текст с засечками
	
	
\begin{document}

%%%%%%% ТТУЛЬНИК %%%%%%%%%
\title{\small{Ответ на экзаменационный \\ билет №6}}
\author{\small{%
\emph{автор:}~Прокудин Борис Сергеевич}}
\institute{Санкт-Петербургский государственный электротехнический университет «ЛЭТИ» им. В.И. Ульянова (Ленина)}
\vspace{30pt}%
%Санкт-Петербургский государственный электротехнический университет\\
%«ЛЭТИ» им. В.И. Ульянова (Ленина)
\vspace{60pt}%

%\date{\small{Санкт-Петербург, 2018}}

\AtBeginSection{
	\begin{frame}
		\frametitle{Содержание}
		\tableofcontents[currentsection]
	\end{frame}
}

\begin{frame}
\titlepage	
\end{frame}




%%%%%%%%%%%%%%%%%%%%%%%%%%%%%%%%%%%%%%%%%%%%%%%%%%%%%%%%%%%%%%
\section{Представление текстовой информации}
\begin{frame}
\frametitle{\small Представление текстовой информации}

Для передачи и хранения информации применяются условные знаки. Самый простой их пример набора таких знаков -- это алфаваит.
\\
\begin{itemize}
\item Код -- совокупность условных знаков.
 
\item Кодирование -- операция преобразования символов одного кода всимволы другого кода.
\end{itemize}


\end{frame}



%%%%%%%%%%%%%%%%%%%%%%%%%%%%%%%%%%%%%%%%%%%%%%%%%%%%%%%%%%%%%%
\begin{frame}
\frametitle{\small Системы кодирования}

Существуют различные системы кодирования. В основном один символ кодируется в один байт.


\small Первой действующей системой кодирования была разработанна в США, и получила название ASCII

\small Код ASCII первоначально использовал 7 бит для представления прописных и строчных букв английского алфавита, цифр, знаков пунктуации и управляющей информации. Управляющая информация была необходима для управления вводом/выводом информации в ЭВМ с перфоленты. В настоящее время код ASCII состоит из 8 бит, одному символу соответствует ровно 1 байт. Это позволило существенно увеличить разнообразие представляемых символов, в том числе добавить символы национальных алфавитов.


\end{frame}
  
  
  %%%%%%%%%%%%%%%%%%%%%%%%%%%%%%%%%%%%%%%%%%%%%%%%%%%%%%%%%%%%%%
\begin{frame}
\frametitle{\small Расширенные кодировки}

Существуют кодировки, в которых один символ занимает количество бит, отличное от восьми.
К таким кодировкам относятся: Unicode и UTF-8.
\begin{itemize}

\item Unicode -- Шестнадцати битная кодировка

\item UTF-8 -- Кодировка использующая переменное количество байт
\end{itemize}

\end{frame}


%%%%%%%%%%%%%%%%%%%%%%%%%%%%%%%%%%%%%%%%%%%%%%%%%%%%%%%%%%%%%%

\begin{frame}
\frametitle{\small Unicode}

Unicode (Юникод) — стандарт кодирования символов, включающий в себя знаки почти всех письменных языков мира. В настоящее время стандарт является доминирующим в Интернете.

\small Стандарт предложен в 1991 году некоммерческой организацией «Консорциум Юникода». Применение этого стандарта позволяет закодировать очень большое число символов из разных систем письменности: в документах, закодированных по стандарту Юникод, могут соседствовать китайские иероглифы, математические символы, буквы греческого алфавита, латиницы и кириллицы, символы музыкальной нотной нотации.







\end{frame}




%%%%%%%%%%%%%%%%%%%%%%%%%%%%%%%%%%%%%%%%%%%%%%%%%%%%%%%%%%%%%%

\begin{frame}
\frametitle{\small UTF-8}


UTF-8 -- распространённый стандарт кодирования текста, позволяющий более компактно хранить и передавать символы Юникода, используя переменное количество байт (от 1 до 4), и обеспечивающий полную обратную совместимость с кодировкой ASCII. Формат UTF-8 был разработан 2 сентября 1992 года Кеном Томпсоном и Робом Пайком

\end{frame}





%%%%%%%%%%%%%%%%%%%%%%%%%%%%%%%%%%%%%%%%%%%%%%%%%%%%%%%%%%%%%%
\section{Табличное задание булевой функции}
\begin{frame}
\frametitle{\small Табличное задание булевой функции}

Таблица истинности — это таблица, описывающая логическую функцию.
  
  
Под «логической функцией» в данном случае понимается функция, у которой значения переменных (параметров функции) и значение самой функции выражают логическую истинность. Например, в двузначной логике они могут принимать значения «истина» либо «ложь» 



\end{frame}

%%%%%%%%%%%%%%%%%%%%%%%%%%%%%%%%%%%%%%%%%%%%%%%%%%%%%%%%%%%%%%

\begin{frame}
\frametitle{\small Табличное задание булевой функции}


\begin{table}[h]
%\renewcommand{\arraystretch}{1.8} % всех высота строк
\renewcommand{\tabcolsep}{0.15cm} % отступ от текста до линии разметки таблицы


\begin{tabular}{|c|c|c|c|c|c|c}
\hline % начертить горизонтальную линию.

\multicolumn{2}{|c|}{Аргументы} & \multicolumn{5}{|c|}{Логические операции} \\
\hline

\rule{0cm}{0.5cm}% Расширяем строку вверх. т.к. не было видно обозначения отрицания (штрихи) 

% Заполняем логические операции чуть меньшим шрифтом
A  & B & 
$ \bar{A} $ & 
$A \land B $ & 
$A \lor B $ & 
$ A \equiv B $ &
$A \to B $ \\
\hline
  
   1 & 1 & 0 & 1 & 1 & 1 & 1  \\
   1 & 0 & 0 & 0 & 1 & 0 & 0  \\
   0 & 1 & 1 & 0 & 1 & 0 & 1  \\
   0 & 0 & 1 & 0 & 0 & 1 & 1  \\
  
\hline
 
\end{tabular}
\end{table} 


\end{frame}

%%%%%%%%%%%%%%%%%%%%%%%%%%%%%%%%%%%%%%%%%%%%%%%%%%%%%%%%%%%%%%

\end{document}

