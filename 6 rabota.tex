\documentclass{article}
\usepackage[T2A]{fontenc}
\usepackage[utf8]{inputenc}
\usepackage[english,russian]{babel}
\usepackage{tikz}
\usepackage[european,cuteinductors,smartlabels]{circuitikz}
\usepackage{siunitx}
\title{Практическая работа №6}
\author{студент: группы 8871 Прокудин Б.С.}
% Конец преамбулы
\begin{document}
\maketitle

\begin{tikzpicture} 
\newcommand{\D}{6}
\foreach \f in {0, 10, 20, 30, 40, 50, 60, 70, 80, 90, 100, 110, 120, 130, 140, 150, 160, 170, 180, 190,200, 210, 220, 230, 240, 250, 260, 270, 280, 290, 300, 310, 320, 330, 340, 350} 
\draw[red,->,>=stealth'] (0,0) -- ({\D*cos(\f)},{\D*sin(\f)}) node[right] {$\f$} -- ++ ({(\D/5.5)*cos((\f-180)+2*(\f))},{(\D/5.5)*sin((\f-180)+2*(\f))}); % F*2 а не на 3 т.к. при команде ++ уже добавляется 1 угол f
\draw[thin,<->] (2,0) arc(0:10:2) node[midway,right] {$\omega$};
\end{tikzpicture}
\end{document}
