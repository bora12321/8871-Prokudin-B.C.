\documentclass[russian,utf8,nocolumnxxxi,nocolumnxxxii]{eskdtext}
\usepackage[T1,T2A]{fontenc}
\usepackage[utf8]{inputenc}
\usepackage{amssymb,amsmath}
\usepackage{float}
\usepackage{tikz}
\usepackage{rotating}
\usepackage{graphicx}
\graphicspath{{pictures/}}
\DeclareGraphicsExtensions{.pdf,.png,.jpg}
\usepackage{pgfplots}
\usepackage{lipsum}
\usepackage{nccmath}
\usepackage{siunitx}
\usepackage[european,cuteinductors,smartlabels]{circuitikz}
\usepackage[backend=biber]{biblatex}

\begin{document}

\\{\bfСодержание}
\\1. Цель и тема курсовой работы
\\2. Задание на курсовую работу
\\3. Введение
\\4. Исследование функции
\\5. Исследование кубического сплайна
\\6. Задача оптимального распределения неоднородных ресурсов
\\7. Список литературы
\newpage

{\large\bf 1. Цель и тема курсовой работы}

{\bfЦель курсовой работы:} уметь применять персональный компьютер и математические пакеты прикладных программ в инженерной деятельности.
\\ {\bfТема курсовой работы:} решение математических задач с использованием математического пакета «SciLab» и системы компьютерной алгебры «Reduce».
\newpage
{\large\bf 2. Задания на курсовую работу}
\newpage

{\bf3. Введение}

В настоящее время при решении различных как прикладных инженерных, так и чисто исследовательских задач, возникает необходимость в использовании широкого круга алгоритмов из множества разделов математики. Между тем самостоятельная реализация многих алгоритмов на некотором языке программирования может быть сложна и избыточна. Вследствие этого широкое распространение получили математические пакеты и системы компьютерной алгебры, такие как: MatLab, Octave, SciLab, Mathematica, Reduce, Mapple, призванные избавить пользователя от рутинных процедур, предоставить удобный интерфейс взаимодействия с уже написанным программным кодом и быстрым созданием нового. К сожалению, некоторые из перечисленных выше математических пакетов, будучи коммерческими по природе, имеют пакетом SciLab и системой компьютерной алгебры Reduce.

\end {document}