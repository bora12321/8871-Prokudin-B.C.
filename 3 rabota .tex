\documentclass[a4paper,11pt]{article}
\usepackage[T2A]{fontenc}
\usepackage[utf8]{inputenc}
\usepackage[english, russian] {babel}
\usepackage{tikz}
\usepackage[european,cuteinductors, smartlabels]{circuitikz}
\title{Практическая работа № 3}
\author{Прокудин Борис Сергеевич}

\begin{document}
 
\maketitle

График функции $x^3+2x^2-5x-6$ с проведёнными касательными. $Y_{k1}=-6.875+17.75(x+3.5)$ и $Y_{k2}=-1.217+11.615(x+3.113)$


\begin{tikzpicture}

% Координатные оси
\draw[thin, ->] (-7,0) -- (7,0) node[right] {$X$};% осьХ
\draw[thin, ->] (0,-8.5) -- (0,6) node[left] {$Y$};% ось У

% 1. Задаём множество с текстом ; 2. изображаем множество и достраеваем штрихи ; 2 строго под 1
\foreach \x\xtext in {-4,-3,-2,-1,1,2,3} % 1
\draw (\x,0.1) -- (\x,-0.1) node[below right] {$\xtext$};% 2. 

% График функции 
\draw[domain=-3.55:2.3, smooth, green] plot ({\x},{(\x^3)+(2*(\x)^2)-(5*\x)-6});

% Первая касательная
\draw[domain=-3.58:-2.8, smooth, help lines, red] plot ({\x},{(17.75*(\x))+55.25});

% Перпендикуляр от точки пересечения первой касательной с осью Х до функции   и перпендикуля
% Для наглядности нахождения точек введём команды
\newcommand {\W} {0}         
\newcommand {\D} {((\W-55.25)/17.75)} 
\draw[thin,dashed] ({(\W-55.25)/17.75},{\W}) -- ({\D},{(\D^3)+(2*(\D)^2)-(5*\D)-6});
% Координата Х точки касания касательной к функции (-55.25)/17.75)= 3.112676
% Координата Y точки касания касательной к функции({\D},{(\D^3)+(2*(\D)^2)-(5*\D)-6})=1.217062342

% Вторая касательная
\draw[domain=-3.65:-2.55, smooth, help lines, blue] plot ({\x},{(11.6156*(\x))+34.9384});

\end{tikzpicture}

% фрагмент. увеличение мест касание касательных
\begin{tikzpicture} [scale=5]
\draw[help lines] (-4,1) grid (-2,-2);

% Координатные оси
\draw[thin, ->] (-4,0) -- (-2,0) node[right] {$X$};% осьХ

% 1. Задаём множество с текстом ; 2. изображаем множество и достраеваем штрихи ; 2 строго под 1
\foreach \x\xtext in {-4,-3,-3.5,-2.5} % 1
\draw (\x,0.1) -- (\x,-0.1) node[below right] {$\xtext$};% 2. 

% График функции. начало и конец графика Y_(-2)=-3.177; Y_1=-2.892
\draw[domain=-3.177:-2.892, smooth, green] plot ({\x},{(\x^3)+(2*(\x)^2)-(5*\x)-6});

% Первая касательная.  Начало и конец графика касательной (-2-55.25)/1775=-3.225;(1-55.25)/1775=-3.056
\draw[domain=-3.225:-3.056,smooth, help lines, red] plot ({\x},{(17.75*(\x))+55.25});

% Вторая касательная. Начало и конец графика касательной (-2-34,94)/11,62=-3.18 ;(-2-34,94)/11,62=-2.921
\draw[domain=-3.18:-2.921, smooth, help lines, blue] plot ({\x},{(11.6156*(\x))+34.9384});

% Перпендикуляр от точки пересечения первой касательной с осью Х до функции   и перпендикуля
% Для наглядности нахождения точек введём команды
\newcommand {\W} {0} % координата Х первой касательной         
\newcommand {\D} {((\W-55.25)/17.75)} % координата Y первой касательной
\draw[thin,dashed] ({(\W-55.25)/17.75},{\W}) -- ({\D},{(\D^3)+(2*(\D)^2)-(5*\D)-6});

\end{tikzpicture}

\end{document}

