\documentclass[a4paper,11pt]{article}
\usepackage[T2A]{fontenc}
\usepackage[utf8]{inputenc}
\usepackage[english, russian] {babel}
\usepackage{tikz}
%\usetikzlibrary{calc} % для мат вычислений, пока не нужен
\usepackage[european,cuteinductors, smartlabels]{circuitikz}
\title{Практическая работа № 3}
\author{Прокудин Борис Сергеевич}

\begin{document}
 
\maketitle

график функции $x^3+2x^2-5x-6$ с проведёнными касательными

\begin{tikzpicture}

% Координатные оси
\draw[thin, ->] (-7,0) -- (7,0) node[right] {$X$};% осьХ
\draw[thin, ->] (0,-8.5) -- (0,6) node[left] {$Y$};% ось У

% 1. Задаём множество с текстом ; 2. изображаем множество и достраеваем штрихи ; 2 строго под 1
\foreach \x\xtext in {-4,-3,-2,-1,1,2,3} % 1
\draw (\x,0.1) -- (\x,-0.1) node[below right] {$\xtext$};% 2. 

% График функции 
\draw[domain=-3.5:2.3, smooth, green] plot ({\x},{((\x)^3)+(2*(\x)*(\x))-(5*(\x))-6});

% попытка сделать касательную
%\draw[domain=-3.5:-3, smooth, red] plot ({\y},{(-6.875)+17.75*((\y)+3.5)};

\end{tikzpicture}
\end{document}

